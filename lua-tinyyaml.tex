\documentclass{l3doc}

\usepackage{listings}


\begin{document}

\title{The tiny YAML Library}

\author{%
  Zeping Lee%
  \thanks{%
    E-mail:
    \href{mailto:zepinglee@gmail.com}
      {zepinglee@gmail.com}%
  }%
}

\date{2025-02-01 v0.4.4}

\maketitle

\begin{documentation}

\section{Introduction}

The \textsf{lua-tinyyaml} package is a lightweight YAML parser written in pure Lua.
It supports a subset of the YAML 1.2 specifications.
This package is a dependency for several other LuaTeX packages, including \textsf{markdown} and \textsf{citeproc-lua}.


\section{Usage}

To parse a YAML string into a Lua table, use the \texttt{parse} function:

\begin{lstlisting}[language={[5.3]Lua}, basicstyle=\ttfamily]
contents = require("tinyyaml").parse(str)
\end{lstlisting}


\section{License}

\texttt{lua-tinyyaml} is distributed under the MIT License.


\end{documentation}

\end{document}
